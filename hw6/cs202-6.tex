\documentclass[12pt]{article}
\usepackage{geometry}
\usepackage{amsmath}
\usepackage{amssymb}
\usepackage{graphicx}
\usepackage{float}
\usepackage{listings,lstautogobble}
\usepackage{bussproofs}

\geometry{margin=1in}

\title{CPSC 202 PSET 6}
\author{CPSC 202 student}
\date{10/26/17 5pm}

\begin{document}
\maketitle

\newcommand{\E}{\mathrm{E}}
\newcommand{\Var}{\mathrm{Var}}
\newcommand{\Cov}{\mathrm{Cov}}


\section*{A.6.1 An oscillating sum}
\begin{enumerate}
  \item[] Give a closed-form expression for $f(n)$ and prove that it is correct.

    My closed form expression for f(n):
    \[ g(n) = 
      \begin{cases}
        \frac{n}{2} & $if n is even$ \\
        \frac{n+1}{2} & $if n is odd$
      \end{cases}
    \]
    Proven by induction on n:
    \begin{enumerate}
      \item[a.] Base case: $n = 0, f(n) = g(n)$
        \begin{itemize}
          \item $f(0) = {(-1)^n \cdot \sum_{k=0}^{0}(-1)^0 \cdot 0} = 0$
          \item $g(0) = \frac{0}{2} = 0$
        \end{itemize}
      \item[b.] Induction case: $f(n) = g(n) \rightarrow f(n+1) = g(n+1)$
        \begin{itemize}
          \item $f(n+1) = (-1)^{n+1} \cdot \sum_{k=0}^{n+1}(-1)^k \cdot k = (-1)^{n+1} \cdot [\sum_{k=0}^{n} (-1)^k \cdot k + (-1)^{n+1}\cdot(n+1)] = (-1)[(-1)^n\cdot \sum_{k=0}^{n}(-1)^k\cdot k + (-1)^{2n-1}\cdot(n+1)]$ \\ where 
            $(-1)^{2n-1} = (-1)^{2n}(-1) = 1(-1) = -1$
          \item[] Therefore, $f(n+1) = (-1)(g(n) - (n+1))$
          \item[C1.] n is even: $f(n+1) = (-1)(\frac{n}{2} - (n+1)) = (-1)(-\frac{n+2}{2}) = \frac{n+2}{2} = \frac{n+1+1}{2} = \frac{n+1}{2} + \frac{1}{2} = \lceil\frac{n+1}{2}\rceil$
          \item[C2.] n is odd: $f(n+1) = (-1)(\frac{n+1}{2} - (n+1)) = (-1)(-\frac{n+1}{2}) = \frac{n+1}{2} = \lceil\frac{n+1}{2}\rceil$
        \end{itemize}
        Therefore, $f(n)=g(n) \rightarrow f(n+1) = g(n+1)$
    \end{enumerate}
\end{enumerate}

\section*{A.6.2 An approximate sum}
\begin{enumerate}
  \item[] Show that $\sum_{k=1}^{n}k^2\cdot2^k = \Theta(n^2\cdot2^n)$
    $\Theta$ means that the function must be $O$ and $\Omega$.
    Find $c \geq 0$ s.t. $N = 1$
    \begin{enumerate}
      \item[a.] $O(f(n)): \exists c,N$ s.t. $g(n) \leq c(f(n)), n \geq N$
        \begin{itemize}
          \item $\sum_{k=1}^{n}2^k\cdot k^2 = 2^1\cdot1^2 + 2^2\cdot2^2 + ... + 2^n\cdot n^2 \geq 2^n\cdot n^2\cdot1$
          \item c = 1
        \end{itemize}
      \item[b.] $\Omega(f(n)): \exists c$ s.t. $g(n) \geq c(f(n)), n\geq N$
        \begin{itemize}
          \item $\sum_{k=1}^{n}k^2\cdot2^k \leq \sum_{k=1}^{n}n^2\cdot2^k = n^2 \cdot \sum_{k=1}^{n}2^k = n^2 \cdot \frac{2-2^{n+1}}{1-2} = n(2^{n+1} -2) \leq n^2\cdot2^{n+1} = n^2\cdot2^n\cdot2$
          \item c = 2
        \end{itemize}
    \end{enumerate}

\end{enumerate}

\section*{A.6.3 A stretched function}
\begin{enumerate}
  \item[] Prove $f(g(n))$ is in $O(n)$
    \begin{itemize}
      \item $f(n) = O(n) \rightarrow \exists c1, N1$ where for $n\geq N1, f(n) \leq c1\cdot n$
      \item $g(n) = O(n) \rightarrow \exists c2, N2$ where for $n\geq N2, g(n) \leq c2\cdot n$
      \item $f(g(n)) = O(n) \rightarrow \exists c3, N3$ where for $n\geq N3, f(g(n))\leq c3\cdot n$
      \item Case 1: If $g(n) > N1$ and $n > N2$
        \begin{itemize}
          \item Then, $(f(g(n)) \leq c1\cdot g(n) \leq c1 \cdot c2 \cdot n (c1\cdot c2 = c3)$ and $N3 \geq N2$
        \end{itemize}
      \item Case 2: If $|g(n) \leq N1|$
        \begin{itemize}
          \item $-N1 \leq |g(n)| \leq N1$
          \item If $g(n) \leq N1$
            \begin{itemize}
              \item $n\geq N3\geq \frac{m}{c1\cdot c2}$ for $n < N3$
              \item $f(g(n)) \leq m = \frac{m}{c1\cdot c2} \cdot c1 \cdot c2 \leq N3 \cdot c1 \cdot c2 \leq c1 \cdot c2 \cdot n$
            \end{itemize}
          \item $f(g(n))$ is bounded, $f(g(n)) < m$
        \end{itemize}
      \item $n \geq \frac{f(g(n)}{c3} \cdot \frac {f(g(n)}{c3} \leq n\cdot c3$ so $n \geq \frac{m}{c3}$
    \end{itemize}
\end{enumerate}

\end{document}
