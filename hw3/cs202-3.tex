\documentclass[12pt]{article}
\usepackage{geometry}
\usepackage{amsmath}
\usepackage{amssymb}
\usepackage{graphicx}
\usepackage{float}
\usepackage{listings,lstautogobble}
\usepackage{bussproofs}
\geometry{margin=1in}

\title{CPSC 202 PSET 3}
\author{CPSC 202 student}
\date{9/27/17 5pm}

\begin{document}
\maketitle

\newcommand{\E}{\mathrm{E}}
\newcommand{\Var}{\mathrm{Var}}
\newcommand{\Cov}{\mathrm{Cov}}

\section*{A.3.1 A powerful problem}
\begin{enumerate}
  \item[] If ${A}$ and ${B}$ are sets, then ${A^B}$ is the set of all functions ${f: B \rightarrow A}$. 1 = ${\{\emptyset\}}$ \\ ${|1^A|}$ = ${|A^1|}$ then ${|A|}$ = 1
    \begin{enumerate}
      \item[a.] for 1, ${|1|}$ is 1 because it contains exactly one set, the null set.
      \item[b.] ${|1^A|}$ is ${|1|^{|A|}}$ and because ${|1|}$ is 1, ${|1|^{|A|}}$ is also 1.
      \item[c.] This means that 1 = ${|A^1|}$, and we simplify ${|A|^{|1|}}$ to just ${|A|}$
      \item[d.] 1 = ${|A|}$
    \end{enumerate}
\end{enumerate}

\section*{A.3.2 A correspondence}
\begin{enumerate}
  \item[]Prove or disprove: For any sets ${A,B,}$ and ${C,}$ there exists a bijective function ${f: C^{A\times B} \rightarrow (C^B)^A}$.
    \begin{enumerate}
      \item[a.] ${F(f) = g \leftrightarrow \forall a\in A, \forall b \in B, \forall c \in C : f(a,b) = c \rightarrow (g(a) = h \wedge h(b) = c)}$
      \item[b.] Prove injective
        \begin{itemize}
          \item F(f) = g and F(f') = g
          \item ${\forall a \forall b \forall c : f(a,b) = c = f'(a, b) \rightarrow f=f'}$
          \item Therefore, f = f'
        \end{itemize}
      \item[c.] Prove surjective by contradiction
        \begin{itemize}
          \item ${\exists g: \forall f : F(f) \neq g}$
          \item ${\forall f: \exists a \exists b \exists c : f(a,b) \neq c \wedge (g(a) = h \wedge h(b) = c)}$
          \item But fs are all functions ${A \times B \rightarrow C}$: contradiction
        \end{itemize}
    \end{enumerate}
\end{enumerate}

\section*{A.3.3 Inverses}
\begin{enumerate}
  \item[]Show that ${f}$ is injective and ${g}$ is surjective.
    \begin{enumerate}
      \item[a.] Suppose we have elements x and y in f(x). If f(x) = f(y) then x = y, for f to be injective. Therefore, if g(f(x)) = g(f(y)), x = y
      \item[b.] For g to be surjective, this means ${\forall x \in A :\exists y \in B}$ s.t. g(y) = x 
        \begin{itemize}
          \item x = g $\circ$ f(x) = g(f(x))
          \item choose y = f(x)
          \item g(y) = x
        \end{itemize}
    \end{enumerate}
\end{enumerate}
\end{document}
