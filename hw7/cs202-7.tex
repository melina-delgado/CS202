\documentclass[12pt]{article}
\usepackage{geometry}
\usepackage{amsmath}
\usepackage{amssymb}
\usepackage{graphicx}
\usepackage{float}
\usepackage{listings,lstautogobble}
\usepackage{bussproofs}

\geometry{margin=1in}

\title{CPSC 202 PSET 7}
\author{CPSC 202 student}
\date{11/1/17 5pm}

\begin{document}
\maketitle

\section*{A.7.1 Divisibility}
\begin{enumerate}
  \item[] Show that $\forall n \in \mathbb{N} : 12 | (n(n+1)(n+2)(n+3))$
    \begin{enumerate}
      \item[a.] If $\forall n \in \mathbb{N} : 12 | (n(n+1)(n+2)(n+3))$ then we can say $\forall n \in \mathbb{Z}_{12} : n(n+1)(n+2)(n+3) = 0$ (mod 12)
      \item[b.] 12 is divisible by 3 and 4. \\
        Because 3 and 4 are relatively prime, by the Chinese Remainder Theorem: \\ 
        $n(n+1)(n+2)(n+3) = 0$ (mod 12) $\iff n(n+1)(n+2)(n+3) = 0$ (mod 3) and $n(n+1)(n+2)(n+3) = 0$ (mod 4).
      \item[c.] In $\mathbb{Z}_3, n = 0, n = 1, n = 2$
        \begin{itemize}
          \item $n=0: 0(0+1)(0+2)(0+3) = 0$ (mod 3)
          \item $n=1: 1(1+1)(1+2)(1+3) = 1(2)(3)(4)$ and in $\mathbb{Z}_3$, $1(2)(0)(1) = 0$ (mod 3) 
          \item $n=2: 2(2+1)(2+2)(2+3) = 2(3)(4)(5)$ and in $\mathbb{Z}_3$, $2(0)(1)(2) = 0$ (mod 3) 
        \end{itemize}
      \item[d.] In $\mathbb{Z}_4, n = 0, n = 1, n = 2, n = 3$
        \begin{itemize}
          \item $n=0: 0(0+1)(0+2)(0+3) = 0$ (mod 3)
          \item $n=1: 1(1+1)(1+2)(1+3) = 1(2)(3)(4)$ and in $\mathbb{Z}_3, 1(2)(3)(0) = 0$ (mod 4)
          \item $n=2: 2(2+1)(2+2)(2+3) = 2(3)(4)(5)$ and in $\mathbb{Z}_3, 2(3)(0)(1) = 0$ (mod 4)
          \item $n=3: 3(3+1)(3+2)(3+3) = 3(4)(5)(6)$ and in $\mathbb{Z}_3, 3(0)(1)(2) = 0$ (mod 4)
        \end{itemize}
    \end{enumerate}
\end{enumerate}

\section*{A.7.2 Squares}
\begin{enumerate}
  \item[] Let p be prime. Show that if $x^2 = y^2$ (mod p) then either $x = y$ (mod p) or $x = -y$ (mod p)
    \begin{enumerate}
      \item[a.] Subtract $y^2$ from both sides: $x^2-y^2 = 0$ (mod p)
      \item[b.] In mod p, $p | x^2-y^2$ and after factoring, $p | (x-y)(x+y)$ since p is prime. 
      \item[c.] In the case that $p | x-y$, $x-y = 0$ (mod p), and $x = y$ (mod p)
      \item[d.] In the case that $p | x+y$, $x+y = 0$  (mod p), and $x = -y$ (mod p)
      \item[e.] Therefore, the statement is true.
    \end{enumerate}
\end{enumerate}

\section*{A.7.3 A Series of Unfortunate Exponents}
\begin{enumerate}
  \item[] If $x_0$ and $k$ are both odd, then $x_{2b^2} = x_0$ 
    \begin{enumerate}
      \item[a.] $x_{i+1} = x_i^k$ is equivalent to
        \[ x_n = 
          \begin{cases}
            x_0, n = 0 \\
            x_{n-1}^k, n > 0
          \end{cases}
        \]
      \item[b.] A general equation for $x_n$: $x_n = x_{n-1}^k = x_0^{k^n}$
        
        Proof that $x_n = x_0^{k^n}$ by induction on $n$.
        \begin{itemize}
          \item Base case: $n=0: x_n = x_0$ and $x_0^{k^n} = x_0^{k^0} = x_0$. Therefore, $x_n = x_0^{k^n}$ when $n=0$
          \item Induction step: $x_n = x_0^{k^n} \rightarrow x_{n+1} = x_0^{k^{n+1}}$
            \begin{itemize}
              \item $x_0^{k^{n+1}} = x_0^{k\cdot k^n} = (x_0^{k^n})^k$ 
              \item $x_{n+1} = x_n^k$ since $n>0$.
              \item For $x_n^k = (x_0^{k^n})^k$, take the k root: $x_n = x_0^{k^n}$
            \end{itemize}
        \end{itemize}
      \item[c.] $n = 2^{b-2}: x_{2^{b-2}} = x_0^{k^{2^{b-2}}} = x_0$ (mod $2^b$) if $x_0$ and k are odd. 
      \item[d.] Euler's theorem: $x^{\phi(m)}$ (mod m) = 1 if gcd(x, m) = 1
      \item[e.] Want: $x_0^{k^{2^{b-2}}} = C^{\phi(2^b)}x_0 = C^{2^{b-1}}x_0 = x_0$ (mod $2^b$)
      \item[f.] $x^{k^{b-2}} = C^{\phi(2b)}x_0 = x_0$ (mod $2^b$)

        $\phi(2^b) = (2-1)(2^{b-1}) = 2^{b-1}$
        
        $\phi(2^{b-1}) = 2^{b-2}$
        
        $k^{2^{b-2}}$ mod($2^{b-2}) = 1$

        $k^{2^{b-2}} = q 2^{b-1} + 1$
      \item[g.] $x_0 = x_0^{q\cdot2^{b-1} +1} = x_0\cdot x_0^{q\cdot2^{b-1}} = x_0\cdot (x_0^q)^{2^{b-1}}$ (mod $2^b$)

        = $x_0(x_0^q)^{\phi(2^b)}$ (mod $2^b$)

        = $x_0(1) = x_0$ (mod $2^b$)
    \end{enumerate}
\end{enumerate}

\end{document}
