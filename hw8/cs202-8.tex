\documentclass[12pt]{article}
\usepackage{geometry}
\usepackage{amsmath}
\usepackage{amssymb}
\usepackage{graphicx}
\usepackage{float}
\usepackage{listings,lstautogobble}
\usepackage{bussproofs}

\geometry{margin=1in}

\title{CPSC 202 PSET 8}
\author{CPSC 202 student}
\date{11/8/17 5pm}

\begin{document}
\maketitle

\section*{A.8.1 Minimal and maximal elements}
\begin{enumerate}
  \item[1.] Proof: There exists a nonempty $R\subseteq \mathcal{P}(\mathbb{N})$ with no maximal elements.  
    \begin{itemize}
      \item Let's say $A_{max}$ is our maximal element in our infinite set, where max is any number which designates the maximal power set. 
      \item However, if we add 1 to max, we achieve a new maximal. Because we can continuously add one, there is no maximal.
    \end{itemize}
  \item[2.] Proof: There exists a nonempty $S\subseteq\mathcal{P}(\mathbb{N})$ with no minimal elements.
    \begin{itemize}
      \item Similar to our first proof, we can assign $A_{min}$ to be our minimal element. Then we can subtract 1 from $A_{min}$ for $A_{min-1}$. $A_{min}$ is no longer the minimal in the case. Because we can keep subtracting one, there is no minimal element.
    \end{itemize}
  \item[3.] Proof: There exists a nonempty $T\subseteq\mathcal{P}(\mathbb{N})$ with no minimal nor maximal elements.
    \\ If we consider the set of even natural numbers, we can take the union of that set with the odd natural numbers. There is no maximal.
    \\ Considering the set of even natural numbers, we take the set difference of $A_{min} = 0$, do this continuously for each next $A_{min}=2,4,..$, proving that there is no minimum as we can remove these infinitely.
\end{enumerate}

\section*{A.8.2 No trailing zeros}
\begin{enumerate}
  \item[] $x\sim y \iff x = 2^ky \vee y = 2^kx$ $k\in \mathbb{N}$
    \begin{enumerate}
      \item[1.] Show that $\sim$ is an equivalence relation:
        \begin{enumerate}
          \item[a.] It's reflexive: $x\sim x \iff x=2^kx\vee x=2^kx, k\in\mathbb{N}$
            \\ If $k=0$, then, $x=2^0x = x=1*x$. Therefore, $x=x$ and $\sim$ is reflexive.
          \item[b.] It's symmetric: $x\sim y \rightarrow y\sim x$
            \\ $x\sim y \iff x = 2^ky \vee y= 2^kx, k\in\mathbb{N}
            \\ y\sim x \iff y = 2^kx \vee x=2^ky, k\in\mathbb{N}$
            Assume $x\sim y$: 
            \begin{itemize}
              \item Case 1: $x=2^ky$
                \\Plug x into $y\sim x$: $y=2^k2^ky \vee 2^ky = 2^ky$
                \\$2^ky = 2^ky$, therefore, $y\sim x$ is true.
              \item Case 2: $y=2^kx$
                \\Plug y into $y\sim x$: $2^kx = 2^kx\vee x = 2^k2^kx$
                \\$2^kx = 2^kx$, therefore, $y\sim x$ is true.
            \end{itemize}
          \item[c.] It's transitive: $x\sim y \wedge y\sim z \rightarrow x\sim z$
            \\ $x=2^ky \vee y=2^kx, k\in\mathbb{N}\equiv x=2^my, m\in\mathbb{Z}$
            \\$y\sim z: y=2^{m'}z
            \\ x = 2^m2^{m'}z=2^{m+m'}z, m+m'\in\mathbb{Z}$
            \\ Therefore, $x\sim z$
        \end{enumerate}
      \item[2.] Show that there is a bijection $f : \mathbb{N}\rightarrow\mathbb{N}/\sim$
        \[ f(x) = \begin{cases} 
            [0]_\sim & x=0 \\
            [2x-1]_\sim & x>0 
          \end{cases}
        \]
        \begin{enumerate}
          \item[a.] Prove injective: $\forall x,y, f(x)=f(y) \rightarrow x=y$
            \\ Case 1: $x=0
            \\ f(x) = [0]_\sim =f(y)=0\rightarrow x=y$
            \\ Case 2: $x>0, y>0$
            \\ $[2x-1]_\sim = [2y-1]_\sim
            \\ 2x-1 = 2^k(2y-1)$
            \\ Assume w/o loss of generality that $k\geq 0$ 
            \\ If $k > 0$, then the equality does not hold. Therefore, k must equal 0.
            \\ $2x-1 = 2y-1 \rightarrow x=y$
        \end{enumerate}
    \end{enumerate}
\end{enumerate}

\section*{A.8.3 Domination}
\begin{enumerate}
  \item[] $f\preceq g \iff \forall x\in\mathbb{R} : f(x) \leq g(x)$
    \begin{enumerate}
      \item[1.] Prove that $\preceq$ is a partial order:
        \begin{enumerate}
          \item[a.] Reflexive:
            \begin{itemize}
              \item $f \preceq f : f(x) = f(x)$
              \item $f(x) \leq f(x)$
              \item Therefore, $f \preceq f$
            \end{itemize}
          \item[b.] Antisymmetric: $f \preceq g \wedge g\preceq f \rightarrow f = g$
            \begin{itemize}
              \item $f(x)\leq g(x) \wedge g(x)\leq f(x) \rightarrow f(x)=g(x)$
            \end{itemize}
          \item[c.] Transitive: $f\preceq g\wedge g\preceq h \rightarrow f\preceq h$
            \begin{itemize}
              \item $f(x)\leq g(x) \wedge g(x) \leq h(x) \rightarrow f(x)\leq h(x) \rightarrow f \leq h$
            \end{itemize}
        \end{enumerate}
      \item[2.] Prove or disprove: $\preceq$ is a total order. Disproof:
        \begin{itemize}
          \item $f(x) = x, g(x) = 5$
          \item For $x < 5, f(x) \leq g(x)$, but for $x > 5, f(x) \geq g(x)$. Therefore, $\preceq$ is not a total order, because $f(x) \leq g(x) \wedge f(x) \geq g(x)$ 
        \end{itemize}
      \item[3.] Prove or disprove $\preceq$ is a lattice: Each pair of elements has a greatest lower bound and a ast upper bound. Proof:
        \begin{itemize}
          \item $f\wedge g$ (greatest lower bound) and $f\vee g$ (least upper bound)
          \item $f\wedge g(x) = min(f(x),g(x))$
          \item $f\vee g(x) = max(f(x),g(x))$
        \end{itemize}
    \end{enumerate}
\end{enumerate}

\end{document}
