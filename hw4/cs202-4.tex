\documentclass[12pt]{article}
\usepackage{geometry}
\usepackage{amsmath}
\usepackage{amssymb}
\usepackage{graphicx}
\usepackage{float}
\usepackage{listings,lstautogobble}
\usepackage{bussproofs}
\geometry{margin=1in}

\title{CPSC 202 PSET 4}
\author{CPSC 202 student}
\date{10/04/17 5pm}

\begin{document}
\maketitle

\newcommand{\E}{\mathrm{E}}
\newcommand{\Var}{\mathrm{Var}}
\newcommand{\Cov}{\mathrm{Cov}}

\section*{A.4.1 Covering a set with itself}
\begin{enumerate}
  \item[] Disprove: For any set ${A}$, and any surjective function ${f: A \rightarrow A, f}$ is bijective. 
    \begin{enumerate}
      \item[a.] If A is an infinite set, where ${\forall x \in A : x \in \mathbb{N}}$
      \item[b.] $f$ is surjective because for all x, f(x) = f(y) where x and y are elements of the natural numbers. Every element in the codomain is covered.
      \item[c.] $f$ is not injective because the codomain may be larger than the domain. This is because the infinite set may not have the same cardinality between the domain and codomain. 
      \item[d.] Therefore, the surjective function is not bijective for any set $A$
    \end{enumerate}
\end{enumerate}

\section*{A.4.2 More inverses}
\begin{enumerate}
  \item[] Let A be a set. Suppose that every function ${f: A \rightarrow A}$ has an inverse function ${f^{-1}}$. How many elements can ${A}$ have?
    \begin{enumerate}
      \item[a.] ${|A|}$ = 0. If f is the empty function, then the function is bijective.
      \item[b.] ${|A|}$ = 1. There is exactly 1 element in the domain and codomain of the function. Therefore, there is a bijection.
      \item[c.] ${|A|}$ = 2 or more. Suppose our function is ${f(x) = x^2}$ and ${A = {1, -1}}$. As proven by A.4.1, the function is not bijective for any set A that includes 1 and -1 with this function. Therefore, any set A larger than 2 is does not have an inverse for every function.
      \item[d.] $A$ can have 0 or 1 elements.
    \end{enumerate}
\end{enumerate}

\section*{A.4.3 Rational and irrational}
\begin{enumerate}
  \item[] Let $q$ and $r$ be rel numbers such that $q$ is rational and $r$ is irrational. Show that there exists a rational ${q'}$ such that ${q < q' < r}$
    \begin{enumerate}
      \item[a.] Archimedian property: for ${0 < x < y : \exists n \in \mathbb{N}, s.t.  nx > y}$ 
      \item[b.] Subtract q from both sides: ${0 < r-q}$ and q' = q + n, where ${n < r-q \in Q}$
      \item[c.] Case 1: ${r-q \geq 2 > 1}$
        \begin{itemize} 
          \item ${r-q > 1}$
          \item ${r > 1 + q > q}$ where ${1 + q = q'}$
        \end{itemize}
      \item[d.] Case 2: ${0 < r - q < 2}$ where ${r \neq 0, n \neq 1}$
        \begin{itemize}
          \item ${n(r-q) > 2}$ where n = Q
          \item q' = q + 2/n
        \end{itemize}
    \end{enumerate}
\end{enumerate}

\end{document}
